%!TEX root = main.tex
\chapter{Camera Models and Calibration}

Computer vision works much like the way human vision works. Light rays emitted
from a light source, such as a lamp or the sun, travels until it hits an
object. Depending on the objects physical properties (light absorption,
reflection and emission spectra). The light re-emitted by that object and
which eventually reaches our eyes, determines which color we perceive the
object has.

A camera is very similar to our eyes; in place of the eyes cornea and lens,
iris and retina, a camera has glass lenses, aperture and imager respectively.
There are of course differences, like the retina resting on a curved,
spherical surface, while an imager being a flat squared plane, the camera
lenses are moved to focus, while the eyes lenses changes shape. But none of
this really matters; what is most important is the geometry of the arrangement
of the components. The geometry and, especially, ray tracing mathematics plays
a large role in computer vision.

This chapter will briefly explain how a camera works, camera intrinsic and
extrinsic matrices, and how the images must be prepared before continuing down
the depth estimation pipeline. For more exhaustive information on these
topics, refer to \emph{O'reilly Learning OpenCV}\ref{open-cv}.

\section{Pinhole Camera Model}

The simplest model is the \textit{Pinhole camera model}. The model describes
the mathematical relationship between the coordinates of a 3D point and its 2D
projection onto the image plane. An ideal pinhole camera is one where the
pinhole only lets inn a single light ray from any point in the scene.  An
example of pinhole a camera is shown in figure \ref{fig:pinhole-camera-model}.

\begin{figure}[h]
  \centering
  \label{fig:pinhole-camera-model}
  \includegraphics[width=\textwidth]{images/Pinhole-camera-model.pdf}
  \caption{A diagram of a pinhole camera showing the important features of the model}
\end{figure}

The \emph{principle point} is the pinhole itself, the point which all the rays
projected into the \emph{image plane} intersect. The \emph{optical axis} is
the axis perpendicular to the image plane that intersects the principle point,
and the \emph{center of projection} (also referred to as the optical center)
is where the optical axis intersects the image plane. \emph{X} (capital X) is
the distance between a point \emph{P} in the 3D scene and the optical axis,
and \emph{x} is the distance between projected point \emph{p} and the center
of projection. \emph{f} is the focal length, the distance between the
principle point and the center of projection, and \emph{Z} is the distance
from the principle point to the point P in the 3D scene.

\begin{figure}[h]
  \centering
  \label{fig:rearranged-pinhole-camera-model}
  \includegraphics[width=\textwidth]{images/rearranged-Pinhole-camera-model.pdf}
  \caption{A diagram of a pinhole camera showing the important features of the model}
\end{figure}

As can be seen in the figure, we can use similar triangles to get $-x =
f\frac{X}{Z}$. To make the math easier, the model is usually rearranged such
that the image and pinhole planes, shown in figure \ref{fig:rearranged-
pinhole-camera-model}. In this arrangement, the pinhole is now the center of
projection, the point which all rays travel towards. The image created by rays
intersecting the image plane is equivalent to the image created in the
previous model, but it is no longer upside down. The similar triangles
relationship $x = f\frac{X}{Z}$ is now more evident.


\section{Calibration}

So far, the model is a bit too simplistic to model a real camera. Most of the
properties are very hard to manufacture to perfectly mimic the pinhole model;
getting the optical axis exactly perpendicular to the image plane requires
much more precise than are used for most cameras \ref{fig:imager-glued},
placing the image plane so that the optical center is exactly on the optical
axis, and creating lenses that behave like the pinhole (letting through
exactly one ray from each point in the scene) is impossible. Additionally,
lenses distort the scene projected onto the image plane by bending the light
rays, at different rates depending on where the ray hits the lens.

To deal with these problems, two informational models are used; a model for
the camera's geometry and a distortion model for the lens. These two models
define the intrinsic parameters of the camera. These intrinsic parameters vary
from camera to camera, even identical camera models, because the manufacturing
process often isn't precise enough.

The camera geometry model defines values for the optical center offsets and
and focal length in both x- and y-axis. The reason the focal length is
defined for the x- and y-axis' is that some imagers use rectangular pixels.

The process of obtaining the correct values for the intrinsics for a camera is
called \emph{calibration}.

\subsection{Basic Projective Geometry}

\subsection{Lens Distortion}

